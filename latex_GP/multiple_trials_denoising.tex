\section{Multiple Trials Setting: How to Construct $S(t)$ in Noisy Environment}\label{sec:multiple_trials_setting}
We consider the following experiment setup. Let $J$ represents the number of trials in our experiment which characterize the set of sample that we collected, ie. realisations of $y(t)$ on the discrete subsets of the interval $ \in [0,T]$, which can be specified by \textbf{random or deterministic sub-sampling}. We assume that each trial has a set of $N^i$ samples and $N^i$ varies over trials for $i = 1,\ldots, J$. Let  $\mathbf{y}^{i}$ and $\mathbf{t}^i$ denote $N^i$-dimensional vectors which represent the $N^i$ observed values in the $i$th trial and the $N_i$ corresponding time points being a subsample of $[0,T]$, respectively. Given that, $\mathbf{y}^{i} := y(\mathbf{t}^i) = \big[y(t_1^i), \ldots, y(t_{N^i}^i) \big]$. \textbf{We remark that it is not ensured that for the same time point $t_0\in [0,T]$, that the value of $y(t_0)$ in trial $i_1$ and a value of $y(t_0)$ in trial $i_2$ are equal since the definition of $y(t)$ in Equation \eqref{eq:signal_noisy_y} includes the error term component.}


The sets of the time points for each trial, $\mathbf{t}^i$, can be specified deterministic or be a realisations of the random variable. Regardless of the assumption on the sampling mechanism, the time points collected in the set $\mathbf{t}^i$ can be missing. Therefore, we may distinguish the complete and incomplete cases for the sampling times $\mathbf{t}^i$ and the deterministic or random sampling framework. In the following section we will consider the simplest case, when the elements of  $\mathbf{t}^i$  are obtained deterministically and are not missing. {\color{red} Set up notation and cases for the frameworks which we will consider later for subsampling}

\subsection{Review of Denoising Approaches}
\begin{enumerate}
\item median filter
\item spline filter
\item smoothing splines
\item L1 trend filter
\item Exponential moving average (EMA) / a weighted moving average (WMA)
\end{enumerate}

%In order to specify the distribution of each $c_k(t)$, we collect $M$ paths of $x(t)$ . Therefore, we have $M$ collections of of points $\mathbf{t}^{(1)} , \ldots, \mathbf{t}^{(M)}$, each $N_i$ dimensional for $i \in \Big\{1,\ldots,M\Big\}$ and by  by $\mathbf{x}^{(i)}$ we denote the values of $x(t)$ collected in the trail $i$ on the points $t \in \mathbf{t}^{(i)}$.  The sets $\mathbf{t}^{(i)}$ can be the same.
%The EMD decomposition on each of the $M$ replications  $x^{(i)}$ gives the following representations
%\begin{equation}
%\mathbf{x}^{(i)} = \sum_{k = 1}^K \mathbf{c}_k^{(i)}+ \mathbf{r}^{(i)}_K
%\end{equation}
%where $\mathbf{c}_k^{(i)}$ is an $N_i$ dimensional vector which represents the observed values of the function $c_k(t)$ at the arguments in $\mathbf{t}^{(i)}$. The same logic applies to the definition of vectors $\mathbf{r}^{(i)}_K $. The vector $\bm{\mu}_k^{(i)}$  corresponds to the values of the functions $\mu_k(t)$ at the arguments in $\mathbf{t}^{(i)}$, that is, $\bm{\mu}_k^{(i)} = \mu_k(\mathbf{\mathbf{t}^{(i)}})$.
