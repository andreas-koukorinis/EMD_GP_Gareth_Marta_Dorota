\section{Brownian Bridge Analogue to  construct IMFs}
GP representation does not ensures itself that the predicted function from a given Gaussian process is IMF , that is, it satisfies (I1)-(I2). Therefore, we explire the following approches

Weiner process is a zero mean non-stationary Gaussian Process with the kernel $K(t,t') = min(t,t')$, that is
\begin{equation}
W(t) \sim \text{GP}(0,K(t,t'))
\end{equation}
The Brownian Bridge for $t \in [0,T]$ is defined as
\begin{equation}
B(t) = W(t) - \frac{t}{T} W(T)
\end{equation}
Therefore, it is also the Gaussian Process which is zero mean and has the covariance kernel equals to
\begin{align*}
\mathbf{Cov}(B(t),B(s)) & = \mathbf{Cov}(W(t),W(s)) - \frac{s}{T} \mathbf{Cov}(W(t),W(T)) - \frac{t}{T} \mathbf{Cov}(W(s),W(T)) + \frac{st}{T^2} \mathbf{Cov}(W(T),W(T))\\
& = K(t,s) - \frac{s}{T} K(t,T) - \frac{t}{T} K(T,s) + \frac{ts}{T^2} K(T,T)\\
&  = min(t,t') - \frac{ts}{T}
\end{align*}
The described process $B(t)$ satisfies that $B(0) = B(T) = 0$. The Brownian bridge which statisfied 
$B(t_0) = a$ and $B(_t1) = b$ is a solution to the following SDE system of equations
\begin{equation}
\begin{cases}
& dB(t) = dW(t) \\
& B(t_0) = a \\
& B(t_1) = b
\end{cases},  \text{ for } t_0 \leq t \leq t_1
\end{equation}
Therefore, the Brownian Bridge

and after calculations results in the form
\begin{equation}
B(t) = a + (b-a)\frac{t}{T} + W(t) - \frac{t}{T} W(T)
\end{equation} 
and therefore it is a Gaussian Process 
\begin{equation}
B(t) \sim \text{GP}\Big( a + (b-a)\frac{t}{T} ,K(t,t') - \frac{tt'}{T} \Big)
\end{equation}
for $0 \leq t \leq T$
%%%  https://math.aalto.fi/reports/a481.pdf
%% https://www.diva-portal.org/smash/get/diva2:233650/FULLTEXT01.pdf
\subsubsection{Brownian Bridge Movement Model}

Let $W(t)$ denote the Brownian Motion such that
\begin{equation}
W(t) \sim \mathcal{GP}, \big(0,k(t,t') \big)
\end{equation}
with the kernel function $k(t,t') =  \min \big\{ t,t'\big\} $ and $t \in [0,T]$. Let us define the sequence of $N$ points $0 \leq t_1 < t_2 < \ldots <t_N \leq < T$ such that we require that the process $W(t)$ had the fixed values at that points $W(t_i) = a_i$ for $ i \in \big\{1,\ldots,N \big\}$.  Theretofore, we are looking for a Gaussian process for $t \in [0,T]$ model of a conditional variable defined as 
\begin{equation}
B(t) := W(t) | W(t_1) = a_1, W(t_2) = a_2, \ldots, ... W(t_N) = a_N
\end{equation} 
Let $\mathbf{W} = \big[W(t_1) , W(t_2) , \ldots, ... W(t_N)  \big]$ and $\mathbf{t} = \big[ t_1, t_2, \ldots, t_N \big]$ be $N$-dimensional vectors.  Since $W(t)$ is a Gaussian Process, the random variable $W(t) | W(t_1) , W(t_2) , \ldots, ... W(t_N) $ is also a Gaussian Process with the conditional mean
\begin{equation}
\mu(t) : = \mathbb{E}_{W(t) | W(t_1) , W(t_2) , \ldots, ... W(t_N) }\big[ W(t) \big] = \mathbf{k} \big( t, \mathbf{t} \big)^T \mathbf{K}(\mathbf{t},\mathbf{t} \big)^{-1} \mathbf{W}
\end{equation}
and the covariance function 
\begin{equation}
\k(t,t') = \mathbb{E}_{W(t) | W(t_1) , W(t_2) , \ldots, ... W(t_N) }\Big[ \big[ W(t) -\mu(t)  \big] \big[W(t')  -\mu(t')  \big] \big] \Big] =k(t,t') -  \mathbf{k} \big( t, \mathbf{t} \big)^T \mathbf{K}(\mathbf{t},\mathbf{t} \big)^{-1}   \mathbf{k} \big( t', \mathbf{t} \big)
\end{equation}

where
\begin{equation}
\mathbf{K}(\mathbf{t},\mathbf{t}): =\begin{bmatrix}
t_1 & t_1 & t_1 & \ldots & t_1 \\
t_1 & t_2 & t_2 & \ldots & t_2 \\
t_1 & t_2 & t_3 & \ldots & t_3 \\
\vdots & \vdots & \ddots & \vdots \\
t_1 & t_2 & t_3 & \vdots & t_N
\end{bmatrix}_{N \times N}
\end{equation}
and 
\begin{equation}
\mathbf{k}(t,\mathbf{t}): =\begin{bmatrix}
\min(t,t_1) \\
\min(t,t_2) \\
\vdots \\
\min(t,t_N)
\end{bmatrix}_{N \times 1}
\end{equation}


\subsection{Symmetric Local Extremas of IMFs} 
On every time internal there is a Brownian bridge or constrained Brownian bridge which starts and end from local extrema which are $x^{min} (t)= -x^{max}(t)$ for $t \in [\tau_i,\tau_{i+1}$
\subsection{Nonsymmetric}

\subsection{Bayesian EMD}
1. Construct a set of functions in Bayesian setting to have a IMF representation with restricted posterior (what needs to be satisfied on maxima and minima and how to ensure it)
2. Analogous of Brownian Bridge IMFs in Bayesian setting

Berger's optimal theory. Books on smoothing