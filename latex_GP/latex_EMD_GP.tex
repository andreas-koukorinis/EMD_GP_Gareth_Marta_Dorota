\documentclass[article,moreauthors,pdftex,10pt,a4paper]{ssrn} 
\usepackage{algpseudocode}
\usepackage{algorithm}
\usepackage{multicol}
\usepackage{placeins}
\usepackage[title]{appendix}
\usepackage{bm}
\usepackage{subcaption}
\usepackage{lscape}
\usepackage{longtable,hhline}
\usepackage{pifont}
\allowdisplaybreaks

\newcommand{\chighlight}[1]{%
\colorbox{red!50}{$\displaystyle#1$}}
\DeclareMathSizes{10}{9}{7}{6}
\DeclareMathOperator*{\tr}{\text{Tr}}
\DeclareMathOperator*{\myvec}{\text{vec}}
\DeclareMathOperator*{\diag}{diag}
\DeclareMathOperator*{\argmin}{argmin}
\DeclareMathOperator*{\argmax}{argmax}
\usepackage{colortbl}%
\newcommand{\myrowcolour}{\rowcolor[gray]{0.925}}


\graphicspath{{figs/} }

%=================================================================
\firstpage{1} 
\makeatletter 
\setcounter{page}{\@firstpage} 
\makeatother 

%% TODO: add date, make smaller names of authors, fix correspondence with cross and corresponding aurthor.
%------------------------------------------------------------------
% The following line should be uncommented if the LaTeX file is uploaded to arXiv.org
%\pdfoutput=1

%=================================================================
% Add packages and commands here. The following packages are loaded in our class file: fontenc, calc, indentfirst, fancyhdr, graphicx, lastpage, ifthen, lineno, float, amsmath, setspace, enumitem, mathpazo, booktabs, titlesec, etoolbox, amsthm, hyphenat, natbib, hyperref, footmisc, geometry, caption, url, mdframed, tabto, soul, multirow, microtype, tikz

%=================================================================
%% Please use the following mathematics environments: Theorem, Lemma, Corollary, Proposition, Characterization, Property, Problem, Example, ExamplesandDefinitions, Hypomanuscript, Remark, Definition
%% For proofs, please use the proof environment (the amsthm package is loaded by the MDPI class).
\graphicspath{{figs/} }
%=================================================================
% Full title of the paper (Capitalized)
\Title{Notes: Empirical Mode Decomposition \& Gaussian Processes}

% Author Orchid ID: enter ID or remove command
%\newcommand{\orcidauthorA}{0000-0000-000-000X} % Add \orcidA{} behind the author's name
%\newcommand{\orcidauthorB}{0000-0000-000-000X} % Add \orcidB{} behind the author's name

% Authors, for the paper (add full first names)
\Author{}

% Authors, for metadata in PDF
\AuthorNames{}

\address{% 
}

% Contact information of the corresponding author
\corres{}



\begin{document}

\begin{abstract}
Extension 1, Estimation: Treat each IMF as a separate Gaussian process and then represent the signal using multi-kernel representation of the Gaussian Process. \\
Extension 2, Forecasting: GP representation does not ensures itself that the predicted function from a given Gaussian process is IMF , that is, it satisfies (I1)-(I2). Therefore, we  explore the formulation of IMFs as an analogue of Brownian Bridge.
\end{abstract}
\tableofcontents



\section{Gaussian Processes and EMD: IMFs as Gaussian Processes with non-stationary kernels}
We treat each IMF as a separate Gaussian process and then represent the signal using multi-kernel representation of the Gaussian Process.

\subsection{The Stochastic Representation by Gaussian Processes}
Let $s(t)$ for $t\in [0,\infty]$ be a continuous signal which is observed on discrete grid of points in the interval $[0,T]$. We consider not so uncommon situation, when the observed values of $s(t)$ are perturbed by some noise component. The perturbation of the true signal can be either deterministic (ie an chaotic system) or stochastic.  In fact, the process which we observe is the following
\begin{equation}\label{eq:signal_noisy_y}
y(t) = x(t) + \epsilon, \text{ for } \epsilon \in \mathcal{N}(0, \sigma^2).
\end{equation}
The signal $y(t)$ is observed at $t =( t_1 < \dots <t_N ) = \{ t_i \}_{i=1:N}$, where the subscripts represent the sampling index times. Therefore, our observation set consists of pairs $\big\{t_n,y_n\big\}$ where $y_n = y(t_n)$ for $t_n \in [0,T]$. 

For EMD to exists, the input signal needs to be approximated by a continuous representation; therefore, the discrete signal $s(t)$ is converted back into a continuous analog signal via a spline representation as in Equation \label{cubic_spl}. Having an continuous representation $S(t)$, which is an approximation of $s(t)$, we define the formulation of $S(t)$ given by EMD into $M$ intrinsic mode functions (IMFs) as follows 
\begin{equation}\label{eq:model_x_EMD}
S(t) = \sum_{m = 1}^M \gamma_m(t) + r(t) = \sum_{m = 1}^M \text{Re}\Big\{ A_m(t)  e^{i \theta_m(t)} \Big\} + r(t).
\end{equation}
where $r(t)$ represents a tendency which does not have much of oscillation and therefore characterize the low frequency tend of $S(t)$. The background on the EMD decomposition is given in Subsection \ref{ssec:EMD_background}.

\subsubsection{Continuous signal $S(t)$ as a Gaussian Process}
Out goal is to obtain a stochastic representation of the continuous signal $S(t)$ by a Gaussian Process. We postulate that each IMFs function, $\gamma_m(t)$, is a Gaussian process 
\begin{equation}\label{eq:model_IMF_GP_k}
\gamma_m(t) \sim \mathcal{GP} \Big(0, k_m(t,t')\Big), 
\end{equation}
where $k_m(t,t')$ is a positive definite covariance kernel which is parametrized by a set of parameters $\Psi_m$.  

Let us assume that we sample $S(t)$, and functions $\gamma_m(t)$ for $m = 1,\ldots, M$ at the $N$ time points $t_1 < \ldots <t_N$. We denote by $\mathbf{t}$ the vector of points $t_n$ for $n = 1,\ldots, N$.  

Therefore, given the observations $\gamma_m(\mathbf{t}) "= \big[ \gamma_m(t_1), \ldots, \gamma_m(t_N) \big]$, we would like to predict the values of $\gamma_m(t)$ at the argument $s$ that is $c_k(s)$, given the collected information in the observation set. Since $c(t)$ is a Gaussian Process, the random variable $\gamma_m(s)| \gamma_m(\mathbf{t}), \mathbf{t}$ is a Gaussian Process with the conditional mean
\begin{equation*}
\mu_m(s):=\mathbb{E}_{\gamma_m(t)|\gamma_m(\mathbf{t}), \mathbf{t}} \big[\gamma_m(s) \big] =  \mathbf{k}_m \big(s,\mathbf{t}\big) \mathbf{K}_m \big(\mathbf{t},\mathbf{t}\big)^{-1} \gamma_m(\mathbf{t})
\end{equation*}
and the conditional covariance matrix given by
\begin{equation*}
\tilde{k}_m(s,s'):= \mathbb{E}_{\gamma_m(t)|\gamma_m(\mathbf{t}), \mathbf{t}} \bigg[(\gamma_m(s) - \mu_m(s))(\gamma_m(s') - \mu_m(s'))\bigg] = k_m \big(s,s'\big) - \mathbf{k}_m\big(s,\mathbf{t}\big) \mathbf{K}_m \big(\mathbf{t},\mathbf{t}\big)^{-1} \mathbf{k}_m \big(\mathbf{t},s'\big) ^T
\end{equation*}
where
\begin{align*}
\mathbf{K}_m (\mathbf{t},\mathbf{t}) := \begin{bmatrix}
k_m(t_1,t_1) & k_m(t_1,t_2)& \cdots & k_m(t_1,t_{N}) \\
k_m(t_2,t_1) & k_m(t_2,t_2)& \cdots & k_m(t_2,t_{N}) \\
\vdots & \vdots & \ddots & \vdots  \\
k_m(t_{N}^{(i)},t_1) & k_m(t_{N},t_2)& \cdots & k_m(t_{N},t_{N}) 
\end{bmatrix}_{ N \times N}
\end{align*}
and
\begin{align*}
\mathbf{k}_m (s,\mathbf{t}) := \begin{bmatrix}
k_m(s,t_1) & k_m(s,t_2)& \cdots & k_m(s,t_{N})
\end{bmatrix}_{ 1 \times N}. 
\end{align*}

\paragraph{Multikernel Representation of $S(t)$}
The tendency component $r(t)$ can be modelled as a Gaussian Process itself or one can assume that $S(t)$ is a Gaussian Process conditioned on $r(t)$, that is
\begin{equation}\label{eq:xt_conditional_rep}
S(t) | r(t) \sim \mathcal{GP} \Big(r(t),  k(t,t')  \Big).
\end{equation}
where $k(t,t')$ is a function of the kernels $k_m(t,t')$ for $m \in \big\{1,\ldots,M \big\}$
These two approaches provide an unconditional and conditional stochastic representation of $S(t)$, respectively, and determine two different estimators of the out-of-sample forecast for $S(t)$.  The later is a more convenient assumption to preserve the monotonicity of the $r(t)$ which is a desired property of a residual function in the decomposition in Equation \eqref{eq:model_x_EMD}.  To ensure the function $r(t)$ to have only single convexity change, $r(t)$ might be extrapolated by a power law which stays monotonic (ie a polynomial up to the second order). Then, the out-of-sample forecast of $S(t)$ would be conditioned on the extrapolation of $r(t)$.  In order to preserve the monotonicity property of the tendency function $r(t)$ in the out-of-sample prediction, the extrapolation from a low order spline representation of $r(t)$, which is deterministic,  is excepted to behaves better than the forecast from a Gaussian Process since the later would most plausibly wiggle around a trend and, consequently, would loose the monotonicity of $r(t)$.  In the following work we would like to guarantee the out-of-sample monotonicity of $r(t)$ obtained by construction in the in-ample set,  and therefore, we chose to work with the conditional representation of $x(t)$ given in Equation \eqref{eq:xt_conditional_rep}.  {\color{red} TODO: derive the properties of these two estimators.}.


Given the Gaussian Process model of the $\gamma_m(t)$ in Equation \eqref{eq:model_IMF_GP_k}, the distribution of $S(t)$ can be formulated as a uniform mixture of Gaussian Processes with different kernels.  If we assume that the processes $\gamma_m(t)$ are independent, then, the stochastic representation of $S(t)$ from Equation \eqref{eq:xt_conditional_rep} can be formulated as follows
\begin{equation}
S(t)|r(t) \sim ~   GP \bigg(r(t); \sum_{m=1}^M k_m(t,t') \bigg) 
\end{equation}
If we denote by $k(t,t') := \sum_{m=1}^M k_m(t,t')$, then predictive distribution of $S(t)$ is given by 
\begin{equation}
\mu(s):= \mathbb{E}_{S(t)| r(t), \mathbf{s},\mathbf{t}} \big[S(\mathbf{s})] =  r(\mathbf{s}) + \sum_{m = 1}^M \mu_m(s)
\end{equation}
and the covariance matrix given by
\begin{equation}
\tilde{k}(s,s'):= \mathbb{E}_{S(t)|S(\mathbf{t}), \mathbf{t}} \bigg[(S_m(s) - \mu(s))(S(s') - \mu(s'))\bigg] = \sum_{m = 1}^M \tilde{k}(s,s') 
\end{equation}

If the processes of $\gamma_m(t)$ are not independent, the Gram matrix of the model for $S(t)$ contain additional elements which provide the correlation structure between different IMFs
\begin{equation}
s(t) |r(t)\sim ~   \mathcal{GP} \bigg(r(t); \sum_{m=1}^M k_m(t,t') + 2\sum_{m_1,m_2=1, m_1<m_2}^M k_{m1,m2}(t,t')\bigg) 
\end{equation}
where $k_{m1,m2}(t,t')$ defines the dependence structure between $\gamma_{m_1}(t)$ and $\gamma_{m_2}(t)$.







\subsection{Estimation of the Static Parameters}
\subsubsection{MLE Estimation of the Static Parameters in Gaussian Processes Models}
In the following subsection we derive the MLE estimator of the vectors of parameters $\varphi_k$ and $\Psi_k$.  Given the model in Equation \eqref{eq:model_IMF_GP_k_noisy}, the loglikelihood of the the observation set $\big\{\mathbf{c}_k, \mathbf{t}\big\}$ is the following 
\begin{equation}\label{eq:IMF_loglik_noisy}
l_k\Big( \mathbf{c}_k, \mathbf{t} , \varphi_k, \Psi_k \Big) = - \frac{N}{2} \log 2 \pi - \frac{1}{2} \log |\mathbf{K}_k + \sigma^2_k \mathbb{I}_N | - \frac{1}{2}\mathbf{v}_k^T \Big(\mathbf{K}_k   + \sigma^2_k \mathbb{I}_N \Big)^{-1} \mathbf{v}_k
\end{equation}
where $\mathbf{v}_k = \mathbf{c}_k - \bm{\mu}_k$ and $\mathbf{K}_k$ denotes a $N \times N$ Gram matrix defined as
\begin{align*}
\mathbf{K}_{k}  := K_{k} (\mathbf{t},\mathbf{t}) = \begin{bmatrix}
K_k (\mathbf{t}^{(1)},\mathbf{t}^{(1)}  )& K_k (\mathbf{t}^{(1)},\mathbf{t}^{(2)}  ) & \cdots & K_k (\mathbf{t}^{(1)},\mathbf{t}^{(M-1)}  ) & K_k (\mathbf{t}^{(1)},\mathbf{t}^{(M)}  ) \\
K_k (\mathbf{t}^{(2)},\mathbf{t}^{(1)}  )& K_k (\mathbf{t}^{(2)},\mathbf{t}^{(2)}  ) & \cdots & K_k (\mathbf{t}^{(2)},\mathbf{t}^{(M-1)}  ) & K_k (\mathbf{t}^{(2)},\mathbf{t}^{(M)}  ) \\
\vdots & \vdots & \ddots & \vdots & \vdots  \\
K_k (\mathbf{t}^{(M-1)},\mathbf{t}^{(1)}  )& K_k (\mathbf{t}^{(M-1)},\mathbf{t}^{(2)}  ) & \cdots & K_k (\mathbf{t}^{(M-1)},\mathbf{t}^{(M-1)}  ) & K_k (\mathbf{t}^{(M-1)},\mathbf{t}^{(M)}  ) \\
K_k (\mathbf{t}^{(M)},\mathbf{t}^{(1)}  )& K_k (\mathbf{t}^{(M)},\mathbf{t}^{(2)}  ) & \cdots & K_k (\mathbf{t}^{(M)},\mathbf{t}^{(M-1)}  ) & K_k (\mathbf{t}^{(M)},\mathbf{t}^{(M)}  ) 
\end{bmatrix}_{N \times N}, 
\end{align*}
If the sets of points $\mathbf{t}^{(i)}$ are the same and equal to $\mathbf{t}^*$, the vector $\mathbf{t}$ is constructed by stacking $\mathbf{t}^*$ by $M$ times. Then the formulation of the likelihood simplifies to 
\begin{equation}
l_k\Big( \mathbf{c}_k, \mathbf{t}^* , \varphi_k, \Psi_k \Big) = - \frac{N}{2} \log 2 \pi - \frac{M}{2} \log |K_k (\mathbf{t}^*,\mathbf{t}^*  ) + \sigma^2_k \mathbb{I}_{N_*} | - \frac{1}{2}\sum_{i = 1}^M \mathbf{v}^{(i) \ T} \Big( K_k (\mathbf{t}^*,\mathbf{t}^*  ) +  + \sigma^2_k \mathbb{I}_{N_*} \Big)^{-1}\mathbf{v}^{(i)} \big) 
\end{equation}
Under the formulation of the loglikelihood in Equation \eqref{eq:IMF_loglik_noisy}, the static parameters of the model in Equation \eqref{eq:model_IMF_GP_k_noisy} can be estimated by solving the system of equations given by
\begin{equation}
\nabla l_k\Big( \mathbf{c}_k, \mathbf{t} , \varphi_k, \Psi_k \Big)  = \mathbf{0}
\end{equation}
where $\nabla l_k\Big( \mathbf{c}_k, \mathbf{t} , \varphi_k, \Psi_k \Big) $ denotes the gradient of the loglikelihood with respect to the vector of static parameters given by
\begin{align*}
& \frac{\partial l_k\Big( \mathbf{c}_k, \mathbf{t} , \varphi_k, \Psi_k \Big)}{\partial \varphi_k} = \frac{1}{2} = \mathbf{c}_k \Big(\mathbf{K}_k + \sigma^2_k \mathbb{I}_N\Big)^{-1} \mathbf{v}_k \frac{\partial \mu_k(\mathbf{t})}{\partial \varphi_k} \\
& \frac{\partial l_k\Big( \mathbf{c}_k, \mathbf{t} , \varphi_k, \Psi_k \Big)}{\partial \Psi_k} = \frac{1}{2} \tr \bigg\{\bigg(\Big(\mathbf{K}_k + \sigma^2_k \mathbb{I}_N\Big)^{-1}  \mathbf{v}_k \mathbf{v}_k^T\Big(\mathbf{K}_k + \sigma^2_k \mathbb{I}_N\Big)^{-1} -\Big(\mathbf{K}_k + \sigma^2_k \mathbb{I}_N\Big)^{-1}  \bigg) \frac{\partial \mathbf{K}_k }{ \partial \Psi_k} \bigg\} \\
& \frac{\partial l_k\Big( \mathbf{c}_k , \mathbf{t} , \varphi_k, \Psi_k \Big)}{\partial \sigma^2_k} = \frac{1}{2} \tr \bigg\{\bigg(\mathbf{K}_k + \sigma^2_k \mathbb{I}_N\Big)^{-1}  \mathbf{v}_k \mathbf{v}_k^T\Big(\mathbf{K}_k + \sigma^2_k \mathbb{I}_N\Big)^{-1} -\Big(\mathbf{K}_k + \sigma^2_k \mathbb{I}_N\Big)^{-1}  \bigg\} \\
\end{align*}

\subsubsection{Kernel Alignment}


\subsubsection{Estimators of the Static Parameters given Splines Formulation of $x(t)$}
\input{IMF_pred_distribution}
\subsection{Multikernel Representation of the Signal}

\subsubsection{Assuming Independence of IMFS }
Given the Gaussian Process model of the $c_k(t)$, the distribution of $x(t)$ can be formulated as a uniform mixture of Gaussian Processes with different kernels.  Again, we can either assume that the observed values are or are not perturbed by a noise. In the following derivation we assume that the model of the $x(t)$ includes additional  term corresponding to the zero mean Gaussian noise with variance $\sigma^2$, that is
\begin{equation}
x(t) = \sum_{k = 1}^K c_k(t) + r_K(t) + \epsilon
\end{equation}
and results in the following distribution of $x(t)$ 
\begin{equation}
x(t) \sim ~   GP \bigg(r_K(t) + \sum_{k=1}^K \mu_k(t); \sum_{k=1}^K K_k(t,t') + \sigma^2 \bigg) 
\end{equation}
The scalar $\sigma^2$ can be estimated by MLE of $x(t)$, given its $M$ realization, $\mathbf{x}^{(i)}$, formed into a vector $\mathbf{x} = \big[\mathbf{x}^{(1)},\ldots, \mathbf{x}^{(M)} \big]$. If we denote by $K(t,t') := \sum_{k=1}^K K_k(t,t')$ a vector operator similarly defined as $K_k(t,t')$  and by $\mu(t) = r_K(t) + \sum_{k=1}^K \mu_k(t)$, then the log-likelihood of the model 
\begin{equation}
l\big( \mathbf{x}, \mathbf{t} , \sigma^2 \big) = - \frac{N}{2} \log 2 \pi - \frac{1}{2} \log |K(\mathbf{t},\mathbf{t}) + \sigma^2 \mathbb{I}_N | - \frac{1}{2}(\mathbf{x} - \mu(\mathbf{t}))^T \Big(K(\mathbf{t},\mathbf{t})   + \sigma^2 \mathbb{I}_N \Big)^{-1} (\mathbf{x} - \mu(\mathbf{t}))
\end{equation}
with corresponding gradient
\begin{equation*}
\frac{\partial l\big( \mathbf{x} , \mathbf{t} , \sigma^2 \big)}{\partial \sigma^2} = \frac{1}{2} \tr \bigg\{\Big(K(\mathbf{t},\mathbf{t}) + \sigma^2 \mathbb{I}_N\Big)^{-1}  (\mathbf{x} - \mu(\mathbf{t}))(\mathbf{x} - \mu(\mathbf{t}))^T\Big(K(\mathbf{t},\mathbf{t})  + \sigma^2 \mathbb{I}_N\Big)^{-1} -\Big(K(\mathbf{t},\mathbf{t})+ \sigma^2 \mathbb{I}_N\Big)^{-1}  \bigg\} 
\end{equation*}
The predictive distribution of $x(t)$ is given by 
\begin{equation*}
\mathbb{E}_{x(t)| \mathbf{t}} \big[x(\mathbf{s})] = \sum_{k = 1}^K \mathbb{E}_{c_k(t)|\mathbf{t}} \big[c_k(\mathbf{s})] 
\end{equation*}
and the covariance matrix given by
\begin{equation*}
\mathbf{Cov}_{x(t)|\mathbf{t}} \big[x(\mathbf{s})]  = \sum_{k = 1}^K\mathbf{Cov}_{c_k(t)|\mathbf{t}} \big[c_k(\mathbf{s})]  + \sigma^2
\end{equation*}


\subsubsection{Correlation of IMFS }
If the GP of $c_k$ are not independent, the Gram matrix of the model for $x(t)$ would contain additional elements which provide the correlation structure between different IMFs
\begin{equation}
x(t) \sim ~   GP \bigg(r_K(t) + \sum_{k=1}^K m_k(t); \sum_{k=1}^K K_k(t,t') + 2\sum_{k_1,k_2=1, k_1<k_2}^K K_{k1,k2}(t,t') + \sigma^2 \bigg) 
\end{equation}
where $K_{k1,k2}(t,t')$ defines the dependence structure between $c_{k_1}(t)$ and $c_{k_2}(t)$

\section{Brownian Bridge Analogue to  construct IMFs}
GP representation does not ensures itself that the predicted function from a given Gaussian process is IMF , that is, it satisfies (I1)-(I2). Therefore, we explire the following approches

Weiner process is a zero mean non-stationary Gaussian Process with the kernel $K(t,t') = min(t,t')$, that is
\begin{equation}
W(t) \sim \text{GP}(0,K(t,t'))
\end{equation}
The Brownian Bridge for $t \in [0,T]$ is defined as
\begin{equation}
B(t) = W(t) - \frac{t}{T} W(T)
\end{equation}
Therefore, it is also the Gaussian Process which is zero mean and has the covariance kernel equals to
\begin{align*}
\mathbf{Cov}(B(t),B(s)) & = \mathbf{Cov}(W(t),W(s)) - \frac{s}{T} \mathbf{Cov}(W(t),W(T)) - \frac{t}{T} \mathbf{Cov}(W(s),W(T)) + \frac{st}{T^2} \mathbf{Cov}(W(T),W(T))\\
& = K(t,s) - \frac{s}{T} K(t,T) - \frac{t}{T} K(T,s) + \frac{ts}{T^2} K(T,T)\\
&  = min(t,t') - \frac{ts}{T}
\end{align*}
The described process $B(t)$ satisfies that $B(0) = B(T) = 0$. The Brownian bridge which statisfied 
$B(0) = a$ and $B(T) = b$ is a solution to the following SDE system of equations
\begin{equation}
\begin{cases}
& dB(t) = dW(t) \\
& B(0) = a \\
& B(T) = b
\end{cases},  \text{ for } 0 \leq t \leq T
\end{equation}
and after calculations results in the form
\begin{equation}
B(t) = a + (b-a)\frac{t}{T} + W(t) - \frac{t}{T} W(T)
\end{equation} 
and therefore it is a Gaussian Process 
\begin{equation}
B(t) \sim \text{GP}\Big( a + (b-a)\frac{t}{T} ,K(t,t') - \frac{tt'}{T} \Big)
\end{equation}
for $0 \leq t \leq T$
%%%  https://math.aalto.fi/reports/a481.pdf
%% https://www.diva-portal.org/smash/get/diva2:233650/FULLTEXT01.pdf


\subsection{Symmetric Local Extremas of IMFs} 
On every time internal there is a Brownian bridge or constrained Brownian bridge which starts and end from local extrema which are $x^{min} (t)= -x^{max}(t)$ for $t \in [\tau_i,\tau_{i+1}$
\subsection{Nonsymmetric}

\subsection{Bayesian EMD}
1. Construct a set of functions in Bayesian setting to have a IMF representation with restricted posterior (what needs to be satisfied on maxima and minima and how to ensure it)
2. Analogous of Brownian Bridge IMFs in Bayesian setting

Berger's optimal theory. Books on smoothing




\end{document}
